% 
% Licensed to the Apache Software Foundation (ASF) under one
% or more contributor license agreements.  See the NOTICE file
% distributed with this work for additional information
% regarding copyright ownership.  The ASF licenses this file
% to you under the Apache License, Version 2.0 (the
% "License"); you may not use this file except in compliance
% with the License.  You may obtain a copy of the License at
% 
%   http://www.apache.org/licenses/LICENSE-2.0
% 
% Unless required by applicable law or agreed to in writing,
% software distributed under the License is distributed on an
% "AS IS" BASIS, WITHOUT WARRANTIES OR CONDITIONS OF ANY
% KIND, either express or implied.  See the License for the
% specific language governing permissions and limitations
% under the License.
% 
\section{Overview}

DUCC is a multi-user, multi-system distributed application.  First-time installation is performed in
two stages:

\begin{itemize}
    \item Single-user installation: This provides single-user, single-system installation for testing,
      and verification. Simple development of small applications on small systems such as laptops or
      office workstations is possible after Single-user Installation.
      
    \item Multi-user installation: This provides secure multi-user capabilities and configuration
      for multi-system clusters.
\end{itemize}

First-time users must perform single-user installation and verification on a single system.  Once
this configuration is working and verified, it is straightforward to upgrade to a multi-user
configuration.

\begin{description}
    \item[Note:] A key feature of DUCC is to run user processes in CGroups in order to guarantee
      each process always has the amount of RAM requested. Total RAM allocation to the managed process
      and its child processes that exceed requested memory will go to swap space.
      Without CGroups a process that exceeds its requested memory size by N\% is killed 
      (default N=5 in ducc.properties), and memory use by child processes is ignored.
\end{description}

DUCC is distributed as a compressed tar file.  The instructions below assume installation from one
of this distribution media.  If building from source, the build creates this file in your svn
trunk/target directory. The distribution file is in the form
\begin{verbatim}
   uima-ducc-[version]-bin.tar.gz
\end{verbatim}
where [version] is the DUCC version; for example, {\em uima-ducc-VERSION-bin.tar.gz} (where VERSION is the current DUCC version).  This document will refer to the distribution
file as the ``$<$distribution.file$>$''.

\section{Software Prerequisites}
\label{sec:install.prerequisites}
Both single and multi-user configurations have the following software pre-requisites:

\begin{itemize}
  \item A userid {\em ducc}, and group {\em ducc}.  User {\em ducc} must the the only member of group {\em ducc}.
  \item Reasonably current Linux.  DUCC has been tested on SLES 10.2, 11.1, and 11.2, and RHEL 6.3.
    
    {\em Note:} On some systems the default {\em user limits}
    for max user processes (ulimit -u) and nfiles (ulimit -n) are defined too
    low for DUCC. The shell login profile for user {\em ducc} should set the
    soft limit for max user processes to be the same as the hard limit
    (ulimit -u `ulimit -Hu`), and
    the nfiles limit raised above 1024 to at least twice the number of user
    processes running on the cluster.

  \item Sufficient swap for the expected workload.  If CGroups are configured this is
    at least 1GB and for larger systems, more.
  \item For CGroups support, libcgroup1-0.37+ on SLES and libcgroup-0.37+ on RHEL.  
  \item DUCC has been tested and run on IBM and Oracle JDK 1.6 and 1.7.
  \item Python 2.x, where 'x' is 4 or greater.  DUCC has not been tested on Python 3.x.
\end{itemize}
  
Multi-user installation has additional requirements:

\begin{itemize}
  \item All systems must have a shared filesystem (such as NFS or GPFS)  and common user space 
  \item Passwordless ssh must be installed for user {\em ducc} on all systems.
  \item Root access is required to install a small setuid-root program on each system.
\end{itemize}
  
In order to build DUCC from source the following software is also required:
\begin{itemize}
    \item A Subversion client, from \url{http://subversion.apache.org/packages.html}.  The
      svn url is \url{https://svn.apache.org/repos/asf/uima/sandbox/uima-ducc/trunk}.
    \item Apache Maven, from \url{http://maven.apache.org/index.html}
\end{itemize}

The DUCC webserver server optionally supports direct ``jconsole'' attach to DUCC job processes.  To install
this, the following is required:
\begin{itemize}
    \item Apache Ant, any reasonably current version.
\end{itemize}
    
To (optionally) build the documentation, the following is also required:
\begin{itemize}
  \item Latex, including the \emph{pdflatex} and \emph{htlatex} packages.  A good place
    to start if you need to install it is \url{https://www.tug.org/texlive/}.
\end{itemize}

More detailed one-time setup instructions for source-level builds via subversion can be found here:
\url{http://uima.apache.org/one-time-setup.html\#svn-setup}

\section{Building from Source}

To build from source, ensure you have
Subversion and Maven installed.  Extract the source from the SVN repository named above. 

Then from your extract directory into
the root directory (usually current-directory>/trunk), and run the command
\begin{verbatim}
    mvn install
\end{verbatim}
or
\begin{verbatim}
    mvn install -Pbuild-duccdocs
\end{verbatim}
if you have LaTeX insalled and wish to do the optional build of documentation.

If this is your first Maven build it may take quite a while as Maven downloads all the
open-source pre-requisites.  (The pre-requisites are stored in the Maven repository, usually
your \$HOME/.m2).

When build is complete, a tarball is placed in your current-directory/trunk/target
directory.

\section{Documentation}
\begin{sloppypar}
After single-user installation, the DUCC documentation is found (in both PDF and HTML format) in the directory 
ducc\_runtime/docs.  As well, the DUCC webserver contains a link to the full documentation on each major page.
The API is documented only via JavaDoc, distributed in the webserver's root directory 
{\tt \duccruntime/webserver/root/doc/api.}  
\end{sloppypar}

If building from source, Maven places the documentation in
\begin{itemize}
    \item {\tt trunk/uima-ducc-duccdocs/target/site} (main documentation), and 
    \item {\tt trunk/target/site/apidocs} (API Javadoc)
\end{itemize}

\section{Single-user  Installation and Verification}

Single-user installation sets up an initial, working configuration on a single system.  
Although any user ID can be used to run DUCC in single-user mode, it is recommended to create user ``ducc''
to avoid confusion. No security is established in single-user mode, and all user processes run as
the DUCC user ID.

Verification submits a very simple UIMA pipeline for execution under DUCC.  Once this is shown to be
working, one may proceed to upgrade to full installation.


\section{Minimal Hardware Requirements for single-user Installation}
\begin{itemize}
    \item One Intel-based or IBM Power-based system.  (More systems may be added during multi-user
      installation, described below.)

    \item 8GB of memory.  16GB or more is preferable for developing and testing applications beyond
      the non-trivial.  

    \item 1GB disk space to hold the DUCC runtime, system logs, and job logs.  More is
      usually needed for larger installations.  
\end{itemize}

Please note: DUCC is intended for scaling out memory-intensive UIMA applications over computing
clusters consisting of multiple nodes with large (16GB-256GB or more) memory.  The minimal
requirements are for initial test and evaluation purposes, but will not be sufficient to run actual
workloads.

\section{Single-user System Installation}
\label{subsec:install.single-user}
    \begin{enumerate}
      \item Expand the distribution file:
\begin{verbatim}
tar -zxf <distribution.file>
\end{verbatim}

        This creates a directory with a name of the form ``apache-uima-ducc-[version]''.
  
        This directory contains the full DUCC runtime which
        you may use ``in place'' but it is highly recommended that you move it
        into a standard location; for example, under ducc's HOME directory:
\begin{verbatim}
mv apache-uima-ducc-[version] /home/ducc/ducc_runtime
\end{verbatim}

        We refer to this directory, regardless of its location, as \duccruntime. For simplicity,
        some of the examples in this document assume it has been moved to /home/ducc/ducc\_runtime.

      \item Change directories into the admin sub-directory of \duccruntime: 
\begin{verbatim}
cd $DUCC_HOME/admin
\end{verbatim}

        \item Run the post-installation script: 
\begin{verbatim}
./ducc_post_install
\end{verbatim}
          If this script fails, correct any problems it identifies and run it again.

          Note that {\em ducc\_post\_install} initializes various default parameters which 
          may be changed later by the system administrator.  Therefore it usually should be
          run only during this first installation step.

        \item If you wish to install jconsole support from the webserver, make sure Apache Ant
          is installed, and run
\begin{verbatim}
./sign_jconsole_jar
\end{verbatim}
          This step may be run at any time if you wish to defer it.

   \end{enumerate}

That's it, DUCC is installed and ready to run. (If errors were displayed during ducc\_post\_install
they must be corrected before continuing.)

The post-installation script performs these tasks:
\begin{enumerate}
    \item Verifies that the correct level of Java and Python are installed and available.
    \item Creates a default nodelist, \duccruntime/resources/ducc.nodes, containing the name of the node you are installing on.
    \item Defines the ``ducc head'' node to be to node you are installing from.
    \item Sets up the default https keystore for the webserver.
    \item Installs the DUCC documentation ``ducc book'' into the DUCC webserver root.
    \item Builds and installs the C program, ``ducc\_ling'', into the default location.
    \item Ensures that the (supplied) ActiveMQ broker is runnable.
\end{enumerate}

\section{Initial System Verification}

Here we start the basic installation, submit a simple UIMA-AS job, verify that it ran, and stop
DUCC.  Once this is confirmed working DUCC is ready to use in an unsecured, single-user mode on a
single system.

To run the verification, issue these commands.
\begin{enumerate}
  \item cd \duccruntime/admin 
  \item ./check\_ducc -s
  
    Examine the output of check\_ducc.  If any errors are shown, correct the errors and rerun
    check\_ducc until there are no errors.  
  \item Finally, start ducc: ./start\_ducc -s
  \end{enumerate}
  
  Start\_ducc will first perform a number of consistency checks.
  It then starts the ActiveMQ broker, the DUCC control processes, and a single DUCC agent on the
  local node.  Note that ``single user mode'' (-s) is specified for this first start.  This inhibits
  the checks on the permissions on ducc\_ling (described \hyperref[sec:duccling]{below}).

  You will see some startup messages similar to the following:

\begin{verbatim}
ENV: Java is configured as: /share/jdk1.6/bin/java
ENV: java full version "1.6.0_14-b08"
ENV: Threading enabled: True
MEM: memory is 15 gB
ENV: system is Linux
allnodes /home/ducc/ducc_runtime/resources/ducc.nodes
Class definition file is ducc.mixed.classes
OK: Class and node definitions validated.
OK: Class configuration checked
Starting broker on ducchead.biz.org
Waiting for broker ..... 0
Waiting for broker ..... 1
ActiveMQ broker is found on configured host and port: ducchead.biz.org:61618
Starting 1 agents
********** Starting agents from file /home/ducc/ducc_runtime/resources/ducc.nodes
Starting warm
Waiting for Completion
ducchead.biz.org Starting rm
     PID 14198
ducchead.biz.org Starting pm
     PID 14223
ducchead.biz.org Starting sm
     PID 14248
ducchead.biz.org Starting or
     PID 14275
ducchead.biz.org Starting ws
     PID 14300
ducchead.biz.org
    ducc_ling OK
    DUCC Agent started PID 14325
All threads returned
\end{verbatim}

  Now open a browser and go to the DUCC webserver's url, http://$<$hostname$>$:42133 where $<$hostname$>$ is
  the name of the host where DUCC is started.  Navigate to the Reservations page via the links in
  the upper-left corner.  You should see the DUCC JobDriver reservation in state
  WaitingForResources.  In a few minutes this should change to Assigned.  (This usually takes 3-4
  minutes in the default configuration.) Now jobs can be submitted.
  
  \begin{center}     
  \cfbox{green}{The hostname and port are configurable by
  the DUCC administrator in ducc.properties}
  \end{center}
  
  To submit a job,
  \begin{enumerate}
    \item \duccruntime/bin/ducc\_submit --specification \duccruntime/examples/simple/1.job
    \end{enumerate}
    
    Open the browser in the DUCC jobs page.  You should see the job progress through a series of
    transitions: Waiting For Driver, Waiting For Services, Waiting For Resources, Initializing, and
    finally, Running.  You'll see the number of work items submitted (15) and the number of work
    items completed grow from 0 to 15.  Finally, the job will move into Completing and then
    Completed..

    Since this example does not specify a log directory DUCC will create a log directory in your HOME directory under 
\begin{verbatim}
$HOME/ducc/logs/job-id
\end{verbatim}

    In this directory, you will find a log for the sample job's JobDriver (JD), JobProcess (JP), and
    a number of other files relating to the job.

    This is a good time to explore the DUCC web pages.  Notice that the job id is a link to a set of
    pages with details about the execution of the job.

    Notice also, in the upper-right corner is a link to the full DUCC documentation, the ``DuccBook''.

    Finally, stop DUCC:
    \begin{enumerate}
      \item cd \duccruntime/admin
      \item./stop\_ducc -a
      \end{enumerate}
      
      Once the system is verified and the sample job completes correctly, proceed to Multi-User
      Installation and Verification to set up multiple-user support and optionally, multi-node
      operation.

\section{Multi-User Installation and Verification}
  
    Multi-user installation consists of these steps over and above single-user installation:
    \begin{enumerate}
        \item Install and configure the setuid-root program, ducc\_ling.  This small program allows DUCC
          jobs to be run as the submitting user rather than user ducc.

        \item Optionally install and configure CGroups.

        \item Optionally update the configuration to include additional nodes.
     \end{enumerate}

     Multi-user installation has these pre-requisites (DUCC will not work on multiple nodes
     unless these steps are taken):
     \begin{itemize}

         \item All systems in the DUCC cluster must have a shared filesystem and shared user space (user
           directories are shared over NFS or an equivalent networked file system, across the systems, and
           user ids and credentials are the same).

         \item Passwordless ssh must be installed for user ducc on all systems. Users do NOT require
           ssh access to the DUCC nodes.
           
         \item Root access is (briefly) required to install a small setuid-root program on each system.
      \end{itemize}

\section{Ducc\_ling Installation}
\label{sec:duccling}
    Before proceeding with this step, please note: 
    \begin{itemize}
        \item This step is required ONLY to install multi-user capabilities.
        \item The sequence operations consisting of {\em chown} and {\em chmod} MUST be performed
          in the exact order given below.  If the {\em chmod} operation is performed before
          the {\em chown} operation, Linux will regress the permissions granted by {\em chmod} 
          and ducc\_ling will be incorrectly installed.
    \end{itemize}

    ducc\_ling is a setuid-root program whose function is to execute user tasks under the identity of
    the user.  This must be installed correctly; incorrect installation can prevent jobs from running as
    their submitters, and in the worse case, can introduce security problems into the system.

    ducc\_ling must be installed on LOCAL disk on every system in the DUCC cluster, to avoid
    shared-filesystem access to it.  The path to ducc\_ling must be the same on each system.  For
    example, one could install it to /local/ducc/bin on local disk on every system.

    To install ducc\_ling (these instructions assume it is installed into /local/ducc/bin):
    As ducc, ensure ducc\_ling is built correctly for your architecture:
    \begin{enumerate}
        \item cd \duccruntime/duccling/src
        \item make clean all
     \end{enumerate}
        
     Now, as root, move ducc\_ling to a secure location and grant authorization to run tasks under
     different users' identities:
     \begin{enumerate}
         \item mkdir /local/ducc
         \item mkdir /local/ducc/bin
         \item chown ducc.ducc /local/ducc
         \item chown ducc.ducc /local/ducc/bin
         \item chmod 700 /local/ducc
         \item chmod 700 /local/ducc/bin
         \item cp \duccruntime/duccling/src/ducc\_ling /local/ducc/bin
         \item chown root.ducc /local/ducc/bin/ducc\_ling
         \item chmod 4750 /local/ducc/bin/ducc\_ling
      \end{enumerate}
         
      Finally, update the configuration to use this ducc\_ling instead of the default ducc\_ling:
      \begin{enumerate}
        \item Edit \duccruntime/resources/ducc.properties and change this line:         
\begin{verbatim}
 ducc.agent.launcher.ducc_spawn_path=${DUCC_HOME}/admin/ducc_ling
\end{verbatim}
          to this line (Using the actual location of the updated ducc\_ling, if different from /local/ducc/bin):
\begin{verbatim}
ducc.agent.launcher.ducc_spawn_path=/local/ducc/bin/ducc_ling
\end{verbatim}
        \end{enumerate}


        What these steps do:
      \begin{enumerate}
          \item The first two step compile ducc\_ling for your current machine architecture and
            operating system level.
          \item The next two steps (mkdir) create directory /local/ducc/bin
          \item The next two steps (chown) set ownership of /local/ducc and /local/ducc/bin to user ducc,
            group ducc
          \item The next two steps (chmod) set permissions for /local/ducc and /local/ducc/bin so only user
            ducc may access the contents of these directories
          \item The copy stop copies the ducc\_ling program created in initial installation into /local/ducc/bin
          \item The next step (chown) sets ownership of /local/ducc/bin/ducc\_ling to root, and
            group ownership to ducc.
          \item The next step (chmod) stablishes the {\em setuid} bit, which allows user ducc to execute ducc\_ling
            with root privileges.
          \item Finally, ducc.properties is updated to point to the new, privileged ducc\_ling.
       \end{enumerate}
          
       If these steps are correctly performed, ONLY user {\em ducc} may use the ducc\_ling program in
       a privileged way. Ducc\_ling contains checks to prevent even user {\em root} from using it for
       privileged operations.

       Ducc\_ling contains the following functions, which the security-conscious may verify by examining
       the source in \duccruntime/duccling.  All sensitive operations are performed only AFTER switching
       userids, to prevent unauthorized root access to the system.
       \begin{itemize}
         \item Changes it's real and effective userid to that of the user invoking the job.
         \item Optionally redirects its stdout and stderr to the DUCC log for the current job.
         \item Optionally redirects its stdio to a port set by the CLI, when a job is submitted.
         \item ``Nice''s itself to a ``worse'' priority than the default, to reduce the chances
           that a runaway DUCC job could monopolize a system.
         \item Optionally sets user limits.
         \item Prints the effective limits for a job to both the user's log, and the DUCC agent's log.
         \item Changes to the user's working directory, as specified by the job.
         \item Optionally establishes LD\_LIBRARY\_PATH 
           for the job from the environment variable
\begin{verbatim}
        DUCC_LD_LIBRARY_PATH
\end{verbatim}
           if set in the DUCC job specification. (Secure Linux systems will
           prevent LD\_LIBRARY\_PATH 
           from being set by a program with root authority, so this is
           done AFTER changing userids).

       \end{itemize}

\section{CGroups Installation and Configuration}
    The steps in this task must be done as user root and user ducc.

    To install and configure CGroups for DUCC:
    \begin{enumerate}
       \item Install the appropriate \hyperref[sec:install.prerequisites]{libcgroup package} at level 0.37
         or above.

       \item Configure /etc/cgconfig.conf as follows:
\begin{verbatim}
   # Mount cgroups
   mount {
      memory = /cgroup;
   }
   # Define cgroup ducc and setup permissions
   group ducc {
    perm {
        task {
           uid = ducc;
        }
        admin {
           uid = ducc;
        }
    }
    memory {}
   }
\end{verbatim}
       \item Start the cgconfig service:
\begin{verbatim}
   service cgconfig start
\end{verbatim}
         
       \item Verify cgconfig service is running by the existence of directory: 
\begin{verbatim}
   /cgroups/ducc
\end{verbatim}

       \item Some versions of libcgroup are buggy. As user ducc, verify that the following command executes with no error:
\begin{verbatim}
   cgcreate -t ducc -a ducc -g memory:ducc/test-cgroups
\end{verbatim}
         
       \item Configure the cgconfig service to start on reboot:
\begin{verbatim}
   chkconfig cgconfig on
\end{verbatim}

       \item Update \hyperref[itm:props-agent.cgroups.enable]{ducc.properties} to enable CGroups.
         Note that if CGroups is not installed, the DUCC Agent will detect this and not attempt to
         use the feature.  It is completely safe to enable CGroups in {\em ducc.properties} on
         machines where it is not installed.
    \end{enumerate}

\section{Set up the full nodelists}
   To add additional nodes to the ducc cluster, DUCC needs to know what nodes to start its Agent
   processes on.  These nodes are listed in the file
\begin{verbatim}
$DUCC_HOME/resources/ducc.nodes.  
\end{verbatim}
   
   During initial installation, this file was initialized with the node DUCC is installed on.
   Additional nodes may be added to the file using a text editor to increase the size of the DUCC
   cluster.
 
\section{Full DUCC Verification}

This is identical to initial verification, with the one difference that the job ``1.job'' should be
submitted as any user other than ducc.  Watch the webserver and check that the job executes
under the correct identity.  Once this completes, DUCC is installed and verified.
 
\section{Enable DUCC webserver login}

    This step is optional.  As shipped, the webserver is disabled for
    logins.  This can be seen by hovering over the Login text located in the
    upper right of most webserver pages: 
\begin{verbatim}
System is configured to disallow logins
\end{verbatim}

    To enable logins, a Java-based authenticator must be plugged-in and the
    login feature must be enabled in the ducc.properties file by the DUCC
    administrator.  Also, ducc\_ling should be properly deployed (see 
    Ducc\_ling Installation section above).
    
    A beta version of a Linux-based authentication plug-in is shipped with DUCC.
    It can be found in the source tree:
\begin{verbatim}
org.apache.uima.ducc.ws.authentication.LinuxAuthenticationManager
\end{verbatim}

    The Linux-based authentication plug-in will attempt to validate webserver
    login requests by appealing to the host OS.  The user who wishes to
    login provides a userid and password to the webserver via https, which
    in-turn are handed-off to the OS for a success/failure reply.
    
    To have the webserver employ the beta Linux-based authentication plug-in,
    the DUCC administrator should perform the following as user ducc:
\begin{verbatim}    
1. edit ducc.properties
2. locate: ducc.ws.login.enabled = false
3. modify: ducc.ws.login.enabled = true
4. save
\end{verbatim}

    Note: The beta Linux-based authentication plug-in has limited testing.
    In particular, it was tested using:
\begin{verbatim}
Red Hat Enterprise Linux Workstation release 6.4 (Santiago)
\end{verbatim}    
    
    Alternatively, you can provide your own authentication plug-in.  To do so:
\begin{verbatim}    
1. author a Java class that implements 
   org.apache.uima.ducc.common.authentication.IAuthenticationManager
2. create a jar file comprising your authentication class
3. put the jar file in a location accessible by the DUCC webserver, such as 
    $DUCC_HOME/lib/authentication
4. put any authentication dependency jar files there as well
5. edit ducc.properties
6. add the following:
   ducc.local.jars = authentication/*
   ducc.authentication.implementer=<your.authenticator.class.Name>
7. locate: ducc.ws.login.enabled = false
8. modify: ducc.ws.login.enabled = true
9. save   
\end{verbatim}    

