% 
% Licensed to the Apache Software Foundation (ASF) under one
% or more contributor license agreements.  See the NOTICE file
% distributed with this work for additional information
% regarding copyright ownership.  The ASF licenses this file
% to you under the Apache License, Version 2.0 (the
% "License"); you may not use this file except in compliance
% with the License.  You may obtain a copy of the License at
% 
%   http://www.apache.org/licenses/LICENSE-2.0
% 
% Unless required by applicable law or agreed to in writing,
% software distributed under the License is distributed on an
% "AS IS" BASIS, WITHOUT WARRANTIES OR CONDITIONS OF ANY
% KIND, either express or implied.  See the License for the
% specific language governing permissions and limitations
% under the License.
% 
% Create well-known link to this spot for HTML version
 \section{Overview}

    This chapter provides an overview of the DUCC process logs and how to interpret the
    entries therin.

    Each of the DUCC ``head node'' processes writes a detailed log of its operation to
    the directory \ducchome/logs.  The logs are managed by Apache log4j.  All logs are
    managed by a single log4j configuration file
\begin{verbatim}
        DUCC_HOME/resources/log4j.xml
\end{verbatim}

    The logs are set to roll after some reaching a given size and the number of generations
    is limited to prevent overrunning disk space.  In general the log level is set to
    provide sufficient diagnostic output to resolve most issues.

    Each DUCC component writes its own log as defined in the following table:

    \begin{tabular} {| l | l |}
       \hline
          Component & Log Name \\
      \hline
      \hline
          Resource Manager & rm.log \\
      \hline
          Service Manager & sm.log \\
      \hline
          Orchestrator & or.log \\
      \hline
          Process Manager & pm.log \\
      \hline
          Agent & {\em [hostname].agent.log } \\
      \hline
    \end{tabular}
    
    Because there may be many agents, the agent log is prefixed with the name of the host for
    each running agent.

\section{Common Log Format}
     
    Timestamp   LOGLEVEL  COMPONENT.sourceFileName method-name Jobid-or-NA text

\section{Resource Manager Log (rm.log)}

    The RM log is designed to show all phases of resource scheduling.  Much of the flow of a job can
    be observed in this log alone.  The following specific information is available and is explained in
    more detail below:
    \begin{itemize}
      \item Bootstrap configuration
      \item Node arrival and missesd heartbeats
      \item Node occupancy
      \item Job arrival and status updates
      \item Calculation of job caps
      \item How-much - fair share 
      \item What-of - host assignment and preemption
      \item Defrag
      \item Internal schedule
      \item Published schedule
    \end{itemize}
    
\subsection{Bootstrap Configuration}
   The RM summarizes its entire configuration when it starts up and prints it to the log to
   provide context for subsequent data and as verification that the RM is configured in the
   way it was though to be.  All the following are fond in the bootstrap section and are mostly
   self-explanatory:

   \begin{itemize}
     \item A pretty-print of the class configuration.  This is the same as produced by the {\em check\_ducc -c -v} 
       command.
     \item A summary of all classes, one per line.  This is a more concise display and is similar to the
       DUCC Classes page in the web server.
     \item A listing of all RM configuration parameters and the environment including things such as the
       version of JAVA, the operating system, etc.
     \item Nodepool occupancy.  As host names are parsed from the {\em ducc.nodes} files, the RM log
       shows exactly which nodepool each node is added to.
   \end{itemize}
   
   The RM logs can wrap quickly under high load in which case this information is lost.

\subsection{Node Arrival and Missed Heartbeats}
\subsubsection{Node Arrival}
    As each node ``checks in'' with the RM a line is printed with details about the node.  Some fields
    are redundant but are produced by different components processing the node arrival and thus serve
    as confirmation that all parts are operating correctly.

    A node arrival entry is of the form:
\begin{verbatim}
    [LOGHEADER] Nodepool: power Host added: power :  bluej290-18   shares  3 total    9:          
                              bluej290-18     3             0             3  48128 <none>
\end{verbatim}
    where the fields mean (f the field isn't described here, the value is not relevent to node arrival):
    \begin{description}
      \item[LOGHEADER] is the log entry header as described above.
      \item[Nodepool:power] The node is added to the ``power'' nodepool
      \item[bluej290-18] This is the name of the node
      \item[shares 3] The number of full shares supported on this machine.
      \item[total 9] This is the total shares in the system after this node arrives.
      \item[48128] This is the memory, in KB, on that host.
    \end{description}

\subsubsection{Missed Heartbeats}
    The DUCC Agents send out regular ``heartbeat'' meessages with current node statistics. These
    messages are used by RM to determine if a node has failed.  If a heartbeat does not arrive
    at the specified time this is noted in the log as a {\em missing} heartbeat. If a specific (configurable) number
    of consecutive heartbeats is missed, the RM marks the node offline and instructs the
    DUCC Orchestrator to purge the shares so they can be rescheduled.

    A missed heartbeat log entry is of the form
\begin{verbatim}
    [LOGHEADER] "*** Missed heartbeat ***" NODENAME count[NN]
\end{verbatim}
    where the fields mean:
    \begin{description}
      \item[LOGHEADER] is the log entry header as described above.
      \item[*** Missed heartbeat ***] Indicates this is a missing heartbeat message.
      \item[NODENAME] This is the name of the (possibly) errant host.
      \item[count[N]] This is the number of CONSECUTIVE missing heartbeats.
    \end{description}

    Note that it is not unusual to miss the occasional heartbeat or two due to general network or system load.
    As soon as a heartbeat is received the count is reset to 0.

    If the number of missing heartbeats exceeds the value {\em ducc.rm.node.stability} configured in
    {\em ducc.properties} the node is marked offline and this message is emitted:
\begin{verbatim}
    HEADER "*** ! Notification of node death:" NODENAME
\end{verbatim}

    If the node recovers and rejoins, the NodeArrives message as described above is emitted.

\subsection{Node Occupancy}
    {\em Node occupancy} describes, for each node, the capacity of the node, the work assigned to
    that node, and the number of open shares on that node.  The RM writes the node occupancy 
    to its log before assignment of every new schedule.  The occupancy can be found under the log header line:
\begin{verbatim}
    [LOGHEADER] Machine occupancy before schedule
\end{verbatim}

    Sample node occupancy follows.  The header is included in the log.
\begin{verbatim}
           Name Order Active Shares Unused Shares Memory (MB) Jobs
--------------- ----- ------------- ------------- ----------- ------ ...
 f1n2.bluej.net    16            16             0      255459 206710 206715 207878 206719 207900 206674 
 f1n4.bluej.net    16             0            16      255459 <none>[16]
 f7n2.bluej.net    16             0            16      255459 <none>[16]
f9n10.bluej.net    16             0            16      255459 <none>[16]
 f6n1.bluej.net    16             0            16      255459 <none>[16]
 f7n1.bluej.net    16             3            13      255459 203408 [13]
 f7n3.bluej.net    16            16             0      255459 206716 207904 206720 206717 207879 206718 
f4n10.bluej.net    16            15             1      255459 209155 208975 209153 209155 209153 [1]
 f7n5.bluej.net    16            16             0      255459 208960 
 f1n3.bluej.net    16            16             0      255459 205608 206695 207906 205609 206693 206683 
 f1n1.bluej.net    16             3            13      255459 208913 [13]
f6n10.bluej.net    16             3            13      255459 208977 [13]
 f6n7.bluej.net    16             0            16      255459 <none>[16]
 f7n6.bluej.net    16            15             1      255459 209155 209151 206701 209155 206699 [1]
\end{verbatim}

    The meaning of each column is:
    \begin{description}
      \item[Name] The host name.
      \item[Order] This is the share order of the node.  The number represents the number of shares
        that can be scheduled on this node. (Recall that an actual process may and usually does
        occupy multiple shares.)
      \item[Active Shares] This is the number of the shares on the node which are scheduled
        for work.
      \item[Unused Shares] This is the number of shares available for new work.
      \item[Memory] This is the real memory capacity of the node, as reported by the node's
        Agent process.
      \item[Jobs] Each entry here is the DUCC-assigned id of a job with process assigned to
        this node.  Each entry corresponds to one process.  If an ID appears more than 
        once the job has more than one process assigned to the node; see for example, the
        node {\bf f6n10.bluej.net} with multiple entries for job {\em 206693}.

        When no work is assigned to the node, the string {\bf $<$none$>$} is displayed.  
        
        When there is a number in brackets, e.g. {\bf [13]} for node {\bf f7n1.bluej.net}, the
        number represents the number of shares available to be shcheduled on the node.
    \end{description}
    
    The current node occupancy can be queried interactively with the command:
\begin{verbatim}
    DUCC_HOME/admin/rm_qoccupancy --console
\end{verbatim}
    
\section{Service  Manager Log (sm.log)}
    To be filled in.

\section{ (Orchestrator Log or.log)}
    To be filled in.

\section{Process Manager Log (pm.log)}
    To be filled in.

\section{Agent log Log (hostname.agent.log)}
    To be filled in.
