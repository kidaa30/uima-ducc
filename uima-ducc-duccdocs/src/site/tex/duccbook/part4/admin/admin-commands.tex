% 
% Licensed to the Apache Software Foundation (ASF) under one
% or more contributor license agreements.  See the NOTICE file
% distributed with this work for additional information
% regarding copyright ownership.  The ASF licenses this file
% to you under the Apache License, Version 2.0 (the
% "License"); you may not use this file except in compliance
% with the License.  You may obtain a copy of the License at
% 
%   http://www.apache.org/licenses/LICENSE-2.0
% 
% Unless required by applicable law or agreed to in writing,
% software distributed under the License is distributed on an
% "AS IS" BASIS, WITHOUT WARRANTIES OR CONDITIONS OF ANY
% KIND, either express or implied.  See the License for the
% specific language governing permissions and limitations
% under the License.
% 

\section{Administrative Commands}

   The administrative commands include a command to start DUCC, one to stop it, and one to 
   verify the configuration and query the state of the cluster.

   Note: The scripting that supports these functions runs (by default) in multi-threaded mode so
   large clusters can be started, stopped, and queried quickly.  If DUCC is running on an older
   system, the threading may not work right, in which case the scripts detect this and run
   single-threaded.  As well, all these commands support a ``--nothreading'' option to manually
   disable the threading.

\subsection{start\_ducc}
\label{subsec:admin.start-ducc}

    \subsubsection{{\em Description}}
    The command \ducchome/admin/start\_ducc is used to start DUCC processes. If run with no parameters
    it takes the following actions:
    \begin{itemize}
      \item Starts the management processes Resource Manager, Orchestrator, Process Manager,      
      Services Manager, and Web Server on the local node (where start\_ducc is executed.       
      \item Starts an agent process on every node named in the default node list. 
    \end{itemize}

    \subsubsection{{\em Usage}}

    \begin{description}
      \item[start\_ducc {[options]}] \hfill \\ 
        If no options are given, all DUCC processes are started, using the default node list, 
        {\em ducc.nodes}. 
      
      \end{description}
      
      \subsubsection{{\em Options: }}
      \begin{description}

        \item[-n, --nodelist {[nodefile] }] \hfill \\
          Start agents on the nodes in the nodefile. Multiple nodefiles may be specified: 
\begin{verbatim}
start\_ducc -n foo.nodes -n bar.nodes -n baz.nodes 
\end{verbatim}
          

        \item[-s, --singleuser] \hfill \\
          Start DUCC in ``single-user'' mode.  This inhibits certain checks that are ususally
          needed running in multi-user mode; specifically, permissions on the ducc\_ling
          utility are not verified.  In this mode all jobs are usually run as whatever user is used
          to start DUCC.

        \item[-c, --component {[component] }] \hfill \\
          Start a specific DUCC component, optionally on a specific node. If the component 
          name is qualified with a nodename, the component is started on that node. To qualify 
          a component name with a destination node, use the notation component@nodename. 
          Multiple components may be specified: 
\begin{verbatim}
start\_ducc -c sm -c pm -c rm -c or@bj22 -c agent@n1 -c agent@n2 
\end{verbatim}
          
          Components include: 
          \begin{description}
            \item[rm] The Resource Manager
            \item[or]The Orchestrator
            \item[pm]The Process Manager
            \item[sm]The Service Manager
            \item[ws]The Web Server
            \item[agent]Node Agents
          \end{description}

          \item[--nothreading] If specified, the command does not run in multi-threaded mode
            even if it is supported on the local platform.

      \end{description}

      \subsubsection{{\em Notes: }}
      A different nodelist may be used to specify where Agents are started. As well multiple node 
      lists may be specified, in which case Agents are started on all the nodes in the multiple node 
      lists. 
      
      To start only agents, run start\_ducc specifying a nodelist explicitly. When started like this, the 
      management daemons are not started unless explicitly requested. 
      
      To start only management processes, run start\_ducc with the -m or --management flags. When 
      started like the the agents are not started unless explicitly requested. 
      
      To start a specific management process, run start\_ducc with the -c component parameter, 
      specify the component that should be started. 
      
      \subsubsection{{\em Examples: }}

      Start some nodes from two different nodelist.  This doesn't start any of the management processes
      but it does insure the ActiveMQ Broker is available.
\begin{verbatim}
        start\_ducc -n foo.nodes -n bar.nodes 
\end{verbatim}
                  
      Start an agent on a specific node: 
\begin{verbatim}
        start\_ducc -c agent@a.specific.node 
\end{verbatim}
      
      Start the webserver on node 'bingle': 
\begin{verbatim}
        start\_ducc -c ws@bingle 
\end{verbatim}

      \subsubsection{{\em Debugging:}}

      Sometimes something will not start and it can be difficult to understand why.  To diagnose, it is
      helpful to know that {\em start\_ducc} is simply a wrapper around a lower-level bit of scripting
      that does the actual work.  That lower-level code can be invoked stand-alone, in which case
      console messages that {\em check\_ducc} will have suppressed are presented to the console.

      The lower-level script is called {\em ducc.py} and accepts the same {\em -c component} flag as
      start\_ducc.  If some component will not start, try running {\em ducc.py -c component} directly.
      It will start in the foreground and usually the cause of the problem becomes evident from
      the console.

      For example, suppose the Resource Manager will not start.  Run the following:
\begin{verbatim}
      ./ducc.py -c rm
\end{verbatim}
      and examine the output.  Use {\em CTL-C} to stop the component when done.
      

\subsection{stop\_ducc}
\label{subsec:admin.stop-ducc}

    \subsubsection{{\em Description:}}
    Stop\_ducc is used to stop DUCC processes. If run with no parameters it takes the following 
    actions:
    \todo Garbled by maven or docbook, update this

    \subsubsection{\em Usage:}

    \begin{description}
      \item[stop\_ducc {[options]}] \hfill \\ 
        If no options are given, help text is presented. At least one option is required, to avoid 
        accidental cluster shutdown. 
      \end{description}
    

      \subsubsection{Options:}
        \begin{description}

          \item[-a --all] \hfill \\
            Stop all the DUCC processes, including agents and management processes. This 
            broadcasts a "shutdown" command to all DUCC processes. Shutdown is normally 
            performed gracefully will all process including job processes given time to save state. 
            All user processes, both jobs and services, are sent shutdown signals. Job and service 
            processes which do not shutdown within a designated grace period are then forcibly 
            terminated with kill -9. 
            
          \item[-n, --nodelist [nodefile]] \hfill \\
            Only the DUCC agents in the designated nodelists are shutdown. The processes are sent 
            kill -INT signals which triggers the Java shutdown hooks and enables graceful shutdown. 
            All user processes on the indicated nodes, both jobs and services, are sent "shutdown" 
            signals and are given a minute to shutdown gracefully. Job and service processes which do 
            not shutdown within a designated grace period are then forcibly terminated with kill -9. 
            
\begin{verbatim}
stop\_ducc -n foo.nodes -n bar.nodes -n baz.nodes 
\end{verbatim}

          \item[-c, --component [component]] \hfill \\
            Stop a specific DUCC component. 

            This may be used to stop an errant management component and subsequently restart it 
            (with start\_ducc). 
            
            This may also be used to stop a specific agent and the job and services processes it is
            managing, without the need to specify a nodelist.  
            
            Examples: 

            Stop agents on nodes n1 and n2:
\begin{verbatim}
stop_ducc -c agent@n1 -c agent@n2 
\end{verbatim}
            
            Stop and restart the rm: 
\begin{verbatim}
stop_ducc -c rm 
start_ducc -c rmc 
\end{verbatim}
            
            Components include: 
            \begin{description}
              \item[rm] The Resource Manager.                 
              \item[or] The Orchestrator.                 
              \item[pm] The Process Manager.                 
              \item[sm] The Service Manager.                 
              \item[ws] The Web Server.                 
              \item[agent\@node] Node Agent on the specified node.
              \end{description}

          \item[--nothreading] If specified, the command does not run in multi-threaded mode
            even if it is supported on the local platform.
              
       \end{description}
            
   \subsubsection{{\em Notes:}}
   Sometimes problems in the network or elsewhere prevent the DUCC components from stopping properly.  The
   {\em check\_ducc} command, described in the following section, contains options to query the
   existance of DUCC processes in the cluster, to forcibly ({\em kill -9}) terminate them, and to
   more gracefully terminate them ({\em kill -INT}).
          


\subsection{check\_ducc}
\label{subsec:admin.check-ducc}
    \subsubsection{{\em Description:}}

    Check\_ducc is used to verify the integrity of the DUCC installation and to find and report on
    DUCC processes. It identifies processes owned by ducc (management processes, agents,
    and job processes), and processes started by DUCC on behalf of users.
    
    Check\_ducc can also be used to clean up errant DUCC processes when stop\_ducc is unable 
    to do so. The difference is that stop\_ducc generally tries more gracefully stop processes. 
    check\_ducc is used as a last resort, or if a fast but graceless shutdown is desired. 
    
    \subsubsection{\em{Usage: }}

        \begin{description} 
          \item[check\_ducc {[options]}]
              If no options are given this is the equivalent of: 
\begin{verbatim}
check_ducc -c -n ../resources/ducc.nodes 
\end{verbatim}
              
              This verifies the integrity of the DUCC installation and searches for all the
              processes owned by user {\em ducc} and started by DUCC on all the nodes in ducc.nodes.
        \end{description}
            
     \subsubsection{\em{Options:}}
         \begin{description}
           \item[-n --nodelist {[nodefile]}]
             Only the nodes specified in the nodefile are searched. The option may be specified 
             multiple times for multiple nodefiles. Note that the "local" node is always checked as well. 
\begin{verbatim}
check_ducc -n nlist1 -n nlist2 
\end{verbatim}
                       
           \item[-c --configuration]
             Verify the \hyperref[sec:ducc.classes]{Resource Manager configuration}.

           \item[-p --pids]               
               Rewrite the PID file. The PID file contains the process ids of all known DUCC 
               management and agent processes. The PID file is normally managed by start\_ducc and 
               stop\_ducc and is stored in the file {\em ducc.pids} in directory {\em ducc\_runtime/state}.
               
               Occasionally the PID file can become partially or fully corrupted; for example, if a DUCC 
               process dies spontaneously. Use check\_ducc -p to search the cluster for processes and 
               refresh the PID file. 
               
            \item[-i, --int] \hfill \\
              Use this to send a shutdown signal ({\em kill -INT}) to all the DUCC processes.  The DUCC processes
              catch this signal, close their resources and exit.  Some resources take some time to close, or in
              case of problems, are unable to close, in which case the DUCC processes will unconditionally exit.

              Sometimes problems in the network or elsewhere prevent {\em check\_ducc -i} from terminating
              the DUCC processes.  In this case, use {\em check\_ducc -k}, described below.

            \item[-k, --kill] \hfill \\
              Use this to forcibly kill a component using kill -9. This should only be used if {\em stop\_ducc}
              or {\em check\_ducc -i} does not work.  

            \item[-s, --single\_user] \hfill \\
              Bypass the multi-user permission checks normally done on the ducc\_ling utility.
              
            \item[--nothreading] If specified, the command does not run in multi-threaded mode
              even if it is supported on the local platform.

            \item[-v, --verboase] \hfill \\
              When specified with {\em -c} to check the configuration, this emits a formatted version
              of the node list showing the full structure of the scheduling classes.
              

           \end{description}               
       
\subsection{vary\_off}
\label{subsec:admin.vary-off}
    \subsubsection{{\em Description:}}

    Vary\_off is used to remove a host from scheduling and to evict the work that is running on it.
    This allows for graceful clearance of a host so the host can be take offline for maintenance,
    or any other purpose (such as sharing the host with other applications.)  The DUCC agent
    is NOT stoppped; use 
    \hyperref[subsec:admin.stop-ducc]{stop\_ducc} to stop the agent.  Reservations are not 
    canceled by {\em vary\_off}.
    
    Only the userid that started DUCC may issue {\em vary\_off}; attempts from other userids
    are rejected.

    \subsubsection{\em{Usage: }}

        \begin{description} 
          \item[vary\_off list-of-hosts]
            The {\em list-of-nodes} is a space delimeted list of host to be removed from
              scheduling in the DUCC cluster.
        \end{description}
            
\subsection{vary\_on}
\label{subsec:admin.vary-on}
    \subsubsection{{\em Description:}}

    Vary\_on is used to restore a host to scheduling by DUCC.  If the agent is still
    alive the host becomes immediately available.  The agent is not started by
    {\em vary\_on}; use use 
    \hyperref[subsec:admin.start-ducc]{start\_ducc} to start the agent if needed.

    Only the userid that started DUCC may issue {\em vary\_on}; attempts from other userids
    are rejected.
    
    \subsubsection{\em{Usage: }}

        \begin{description} 
          \item[vary\_on list-of-hosts]
            The {\em list-of-nodes} is a space delimeted list of host to be restored for
              scheduling in the DUCC cluster.
        \end{description}
            
       
